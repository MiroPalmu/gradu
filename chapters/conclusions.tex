\chapter{Conclusions}

In this thesis we introduced the required background theory
of plasma physics and finite-difference methods,
showcased different parts of the PIC method,
derived the growth rate of filamentation instability
in the cold-fuild limit,
and demonstrated the ability of the GPU-accelerated Runko to simulate
the filamentation instabilitiy correctly.

This thesis concludes the first stage of porting Runko to GPUs,
in which only the minimal viable set of features were ported
without focusing on the performance.
Due to the differences between CPUs and GPUs,
significant parts of the code did not translate well to GPUs and were rewritten entirely.
Before speding resources on performance it was a goal to show that
the GPU port is possible and that it gives correct results.

In the future, focus will be more on
the performance and how to effectively utilize modern supercomputers.
Effective utilization of supercomputers would enables Runko
to performe simulations which push the boundaries of computational astrophysics.
Currently the main performance bottleneck is in the particle communication.
Fixing it requires redesining of how the particles are handleded.
Potential option is to move to tag based particle handling
as shown in~\cite{hakobyan_entity_2025}.
Followup work is required to compare the performance of GPU-accelerated Runko
to other PIC codes.

In addition to the performance optimizations,
there will be more features added to the GPU-accelerated Runko.
Currently, its feature set is limited compared to the original one.
For example, dynamically injecting particles to a running simulation
and user-defined boundary conditions are not yet possible with the GPU-accelerated Runko.
Additionally other field propagation, field interpolation, particle pusher,
and current deposition methods can be implemented when needed.
Going beyond PIC,
there are other phsyics which can be added to the PIC simulation.
One avenue is to implement possibility of running PIC in a curved spacetimes
(e.g., Schwarzschild or Kerr).
Other avenue is to implement quantum electrodynamics effects on GPUs
which were already present on the CPU version of Runko~\cite{nattila_radiative_2024}.


%%% Local Variables:
%%% mode: LaTeX
%%% TeX-master: "../gradu"
%%% End:
