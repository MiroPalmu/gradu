\chapter{Conclusions}

In this thesis,
we introduced the necessary background theory of plasma physics
and finite‑difference methods,
presented the key components of the PIC method,
derived the growth rate of the filamentation instability in the cold-fluid limit,
and demonstrated the ability of the GPU‑accelerated Runko
to simulate the filamentation instability correctly.

This thesis concludes the first stage of porting Runko to GPUs,
in which only the minimal viable set of features was ported,
without yet focusing on performance.
Due to the architectural differences between CPUs and GPUs,
significant parts of the code did not translate well and had to be rewritten entirely.
Before investing resources in performance optimization,
the goal was to show that the GPU port is feasible
and that it produces correct physical results.

In the future,
the focus will shift more toward performance and how to effectively utilize
modern supercomputers.
Efficient use of supercomputers would enable Runko to perform simulations
that push the boundaries of computational astrophysics.
Currently, the main performance bottleneck lies in particle communication.
Addressing this will require redesigning how particles are handled.
One potential option is to adopt tag‑based particle handling,
as demonstrated in~\cite{hakobyan_entity_2025}.
Further work is also required to compare the performance of the GPU-accelerated Runko
with other PIC codes.

In addition to performance optimizations,
new features will be added to the GPU-accelerated Runko.
Currently, its feature set is limited compared to the original version.
For example, dynamically injecting particles into a running simulation
and using user‑defined boundary conditions are not yet supported by the GPU implementation.
Furthermore, alternative field‑propagation schemes, field‑interpolation methods,
particle pushers, and current‑deposition algorithms can be implemented when needed.

Beyond the standard PIC framework,
additional physics modules can also be incorporated into the GPU version.
One possible direction is to implement the capability to run
PIC simulations in curved spacetimes (e.g., Schwarzschild or Kerr geometries).
Another avenue is to port existing quantum-electrodynamics
effects--already available in the CPU version of Runko~\cite{nattila_radiative_2024}--
to GPUs.


%%% Local Variables:
%%% mode: LaTeX
%%% TeX-master: "../gradu"
%%% End:
