\chapter{Electron Beam-Plasma Instabilities}

\begin{todo}
  Add references from \cite{bret_multidimensional_2010} intro
\end{todo}

Beam-plasma system consists of two homogeneous counter-streaming pair plasma populations.
In this chapter we will give breaf overview of instabilities present in such system,
derive the maximum growth rates in cold-fluid limit and show that these match
to the measured growth rates from symmetric beam-plasma simulations done with RUNKO.
In the cold-fluid limit and for symmetric beams
the fastest growing instability is so called filamentation instability,
which unstable modes are perturbations in both $\mathbf{E}$ and $\mathbf{B}$.
This means that matching growth rates demonstrates correctness in RUNKO's
ability to probe the full set of Vlasov-Maxwell equations.

\section{Overview}

Beam-plasma systems have been studied since Langmuir first suggested in 1925,
the existence of oscillations in beam-blasma systems.
Later in 1948 Pierce explained Langmuir's observations by demonstrating
unstable oscillations can arise in such systems.
This prompted Bohm and Gross to develope the kinetic theory of ''two-stream instability'',
which contains unstable waves propagating along the beam direction.
It was Fried in 1959 who showed that there exists a second kind of instability,
in which electromagnetic perturbations perpendicular to the flow are unstable.
These modes tend to break up an initially homogeneus beam profile into small-scale current filaments,
which leads to the name ''filamentation instability''.
Fried's article mentioned closely related work of Weible,
who demonstrated the instability of an anistropic two-temperature Maxwellian plasma.
In literature these two instabilities have become almost interchangable,
even though they are technically not equivalent.
In this theses we will use the term ''filamentation instability'' for clarity.

Additionally to the parallel two-stream modes and perpendicular filamentation modes,
the stability of oblique modes can be investigated.
This was proptly done in the cold-fluid limit and it was found that
the unstable spectrum is two-dimensional as the oblique perturbations are likely to be unstable.
After this many temperature-dependent investigations of the 2D spectrum have been done.
Comprehensive kinetic treatments have managed to provide
a unified vision of the entire unstable spectrum
including temperature effects.

\section{Moments of Vlasov equation in cold-fluid limit}

Derivation of the beam-plasma instabilities in the cold-fluid limit
uses moments of the Vlasov eqaution. We will derive them here.

Generally moment $M(\mathbf{p})$ of Vlasov equation is:
\begin{align}
\int M \left(
\partial_t f
+ \mathbf{v} \cdot \nabla f
+ q ( \mathbf{E} + \boldsymbol{\beta} \times \mathbf{B}) \cdot \nabla_\mathbf{p} f = 0
\right) d\mathbf{p}
\end{align}
In the first two terms the derivatives can be trivially taken out of the integrant,
but the third requires some manipulation.

For brievity we'll define $a^i = qE^i + \frac{q}{c}\epsilon^i_{lk}v^l B^k$.
Using the identity $ \partial_{p_i} v^l = \frac{\delta^l_i}{m \gamma} -  \frac{p^lp_i}{m^3c^2\gamma^3}$
we get that $\partial_{p_i}a^i = 0$.
After integration by parts and noting
that physical distributions vanish at infinity, we can write the third term as:
\begin{align}
\int M \mathbf{a} \cdot \nabla_\mathbf{p} f d\mathbf{p}
&= - \int a^i \partial_{p_i} M f d\mathbf{p}
\end{align}
Overall we get that:
\begin{align}\label{eq:general-moment-of-vlasov}
  \partial_t \int M f_\alpha d\mathbf{p}
+ \partial_i \int M v^i f_\alpha d\mathbf{p}
= \int a^i \partial_{p_i} M f_\alpha d\mathbf{p}
\end{align}

A cold-fluid is characterized by a distribution function
$f(\mathbf{x}, \mathbf{p}, t) = n(\mathbf{x}, t)\delta(\mathbf{p} - \mathbf{P}(\mathbf{x}, t))$,
where $\mathbf{P}(\mathbf{x}, t) = m\gamma\mathbf{V}(\mathbf{x}, t)$
is calculated from the bulk velocity $\mathbf{V}(\mathbf{x}, t)$ of the fluid
and $n(\mathbf{x}, t)$ is its fuilds number density.
Inserting this and $M = 1$ to \ref{eq:general-moment-of-vlasov} we obtain continuity equation:
\begin{align}
\label{eq:continuity-in-cold-limit}
\partial_t n + \partial_i ( P^i n_\alpha ) = 0
\end{align}
For $M = \mathbf{p}$ and using the continuity equation we get:
\begin{align}
\partial_t \mathbf{P} + (\mathbf{V} \cdot \nabla) \mathbf{P}
&= q (\mathbf{E} + c^{-1}\mathbf{V} \times \mathbf{B})
\end{align}


$N$ species of charge $q_j$, mass $m_j$, number density $n_j$ and mean velocity $\mathbf{v}_j$.
The number densities are measured in the laboratory frame and not in the proper frame
of the related species. Following shows the key points of a standard derivation
which details can be found in textbook
\cite{bret_multidimensional_2010,infeld_basic_1975,rowlands_electrodynamics_1970,rosenbluth_handbook_1983}.

Initially system is quasi-neutral $\sum_j q_j n_j = 0$ and without a current
$\sum_j q_j n_j\mathbf{v}_j = 0$. It is also assumed that there is no initial
electromagnetic fields.

\section{Numerical validation}

%%% local Variables:
%%% mode: LaTeX
%%% TeX-master: "../gradu"
%%% End:
