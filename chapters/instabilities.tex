\chapter{Electron Beam-Plasma Instabilities}

\begin{todo}
  Add references from \cite{bret_multidimensional_2010} intro
\end{todo}

Beam-plasma system consists of two homogeneous counter-streaming pair plasma populations.
In this chapter we will give breaf overview of instabilities present in such system,
derive the maximum growth rates in cold-fluid limit and show that these match
to the measured growth rates from symmetric beam-plasma simulations done with RUNKO.
In the cold-fluid limit and for symmetric beams
the fastest growing instability is so called filamentation instability,
which unstable modes are perturbations in both $\mathbf{E}$ and $\mathbf{B}$.
This means that matching growth rates demonstrates correctness in RUNKO's
ability to probe the full set of Vlasov-Maxwell equations.

\section{Overview}

Beam-plasma systems have been studied since Langmuir first suggested in 1925,
the existence of oscillations in beam-blasma systems.
Later in 1948 Pierce explained Langmuir's observations by demonstrating
unstable oscillations can arise in such systems.
This prompted Bohm and Gross to develope the kinetic theory of ''two-stream instability'',
which contains unstable waves propagating along the beam direction.
It was Fried in 1959 who showed that there exists a second kind of instability,
in which electromagnetic perturbations perpendicular to the flow are unstable.
These modes tend to break up an initially homogeneus beam profile into small-scale current filaments,
which leads to the name ''filamentation instability''.
Fried's article mentioned closely related work of Weible,
who demonstrated the instability of an anistropic two-temperature Maxwellian plasma.
In literature these two instabilities have become almost interchangable,
even though they are technically not equivalent.
In this theses we will use the term ''filamentation instability'' for clarity.

Additionally to the parallel two-stream modes and perpendicular filamentation modes,
the stability of oblique modes can be investigated.
This was proptly done in the cold-fluid limit and it was found that
the unstable spectrum is two-dimensional as the oblique perturbations are likely to be unstable.
After this many temperature-dependent investigations of the 2D spectrum have been done.
Comprehensive kinetic treatments have managed to provide
a unified vision of the entire unstable spectrum
including temperature effects.

\section{Growth rate in the cold-fluid limit}

As stated at the beginning of the chapter our system consists of two homogeneous
counter-streaming pair plasma populations. Ions are concidered as a fixed neutralizing background.
We’ll call the two populations the beam and the background plasma and
denote their number densities as $n_b$ and $n_p$.
Similarly we’ll denote their bulk velocities as $\ve V_b$ and $\ve V_p$.
Initially system is current and charge neutral:

\begin{align}
  \label{eq:current-neutrality}
  n_{b0}\ve V_{b0} = n_{p0}\ve V_{p0}
\end{align}

and without electromagnetic fields. We’ll define that the beam is along the x-axis.
Without loss of generality we can assume in the following discussion
that the wave vector is in the xy-plane.
Following symbolic calculator technique from \cite{bret_beam-plasma_2007}
we’ll solve $\ve V_{b1}$ and $\ve V_{p1}$ from the linearized momentum transport equation
\eqref{eq:linearized-cold-fluid-momentum-transport} and insert them to the linearized
continuity equation \eqref{eq:linearized-cold-fluid-continuity-eq} to give us $n_{b1}$ and $n_{p1}$.
Likewise these are inserted to the linearized wave equation \eqref{eq:linearized-wave-eq}.
The resulting equation can be manipulated to a matrix form:
\begin{align}
  \label{eq:dispersion-eq-vec}
  \ve T(\ve k, \omega ) \hat{\ve E}_1 = 0
\end{align}
where $\ve T(\ve k, \omega ) \hat{\ve E}_1$ is a matrix-vector product between the matrix
$\ve T(\ve k, \omega )$ and the electric field perturbation.
The matric has following form:
\begin{align}
  \ve T(\ve k, \omega ) = \frac{\omega^2}{c^2} \boldsymbol{\epsilon}(\ve k, \omega)
  + \ve k \otimes \ve k + k^2 \mathbb{I}
\end{align}
where $\mathbb{I}$ is the unit matrix and $\boldsymbol{\epsilon}(\ve k, \omega)$
is the dielectric tensor:
\begin{align}
  \ve T(\ve k, \omega) =
  \begin{bmatrix}
\frac{\omega^2}{c^2}\epsilon_{xx} - k_z^2 & 0 & \frac{\omega^2}{c^2}\epsilon_{xz} + k_z k_x \\
0 & \frac{\omega^2}{c^2}\epsilon_{yy} - k^2 & 0 \\
\frac{\omega^2}{c^2}\epsilon_{xz} + k_x k_z & 0 & \frac{\omega^2}{c^2}\epsilon_{zz} - k_x^2
\end{bmatrix}
\end{align}
Equation \eqref{eq:dispersion-eq-vec} admits solutions only when determinant of
$\ve T(\ve k, \omega)$ vanishes which results in two branches:
\begin{align}
  \epsilon_{yy} = \frac{c^2k^2}{\omega^2}
\end{align}
and
\begin{align}
  \label{eq:dispersion-eq}
  \left( \frac{\omega^2}{c^2}\epsilon_{xx} - k_z^2 \right)
  \left( \frac{\omega^2}{c^2}\epsilon_{zz} - k_x^2 \right)
  = \left( \frac{\omega^2}{c^2}\epsilon_{xz} + k_x k_z \right)^2
\end{align}

For the instabilities the first branch is not interesting as the second holds
solutions the two-stream, the oblique and the filamentation instabilities.
To investigate the nature of the solutions
we'll first define standard dimensionless variables:
\begin{align}
  x = \frac{\omega}{\omega_{pp}} \qquad
  \ve Z = \frac{\ve k V_{b0}}{\omega_{pp}} \qquad
  \alpha = \frac{n_{b0}}{n_{p0}}
\end{align}
Now $x$ can be solved numerically from dispersion relation \eqref{eq:dispersion-eq}.
Figure \ref{fig:perp-growth-rates} plots the normalized growth rate (i.e. imaginary part of $x$)
as a function of $\ve Z$ for two different cases.
For the diluted case ($\alpha \ll 1$) we can see that the largest growth rate
is the oblique angle regime while for the symmetric case ($\alpha = 1$)
the largest growth is at $Z_z = 0$ and $Z_x \rightarrow \infty$
corresponding to the filamentation instability.

\begin{figure}
  \centering
  \includegraphics{chapters/growth-rate.pdf}
  \caption{Growt rates from dispersion relation \eqref{eq:dispersion-eq} for $\gamma_b = 3$.
    Left is the symmetric case $\alpha = 1$ and right is the diluted case $\alpha = 0.1$. }
  \label{fig:perp-growth-rates}
\end{figure}

For the transverse case $k_z = Z_z = 0$ the components of dielectric tensor
$\boldsymbol{\epsilon}(\ve k, \omega)$ simplify to:
\begin{align}
  \epsilon_{xx} &= 1 - \frac{\alpha}{\gamma_b x^2} - \frac{1}{\gamma_p x^2}, \\
  \epsilon_{yy} &= 1 - \frac{\alpha}{\gamma_b x^2} - \frac{1}{\gamma_p x^2}, \\
  \epsilon_{zz} &= 1 + \frac{Z_x x^2 \alpha^2}{\gamma_p x^4} - \frac{Z_x x^2 \alpha}{\gamma_b x^4}
                  - \frac{\alpha}{\gamma_b^3 x^2} - \frac{1}{\gamma_p^3 x^2}, \\
  \epsilon_{xz} &= \frac{Z_x \alpha}{x^3} \left( \frac 1\gamma_p - \frac 1\gamma_b \right)
\end{align}
Inserting these to the dispersion relation \eqref{eq:dispersion-eq}
and taking the limit $Z_x \rightarrow \infty$ the growth rates for
symmetric and diluted cases of filamentation instability can be solved exactly:
\begin{align}
  \label{eq:filamentation-growth-rate-diluted}
  \Im (x) &= \frac{v_b}{c}\sqrt{\frac{\alpha}{\gamma_b}}, \quad \alpha \ll 1 \\
  \label{eq:filamentation-growth-rate-symmetric}
  &= \frac{v_b}{c}\sqrt{\frac{2}{\gamma_b}}, \quad \alpha = 1
\end{align}

\section{Numerical validation}

%%% local Variables:
%%% mode: LaTeX
%%% TeX-master: "../gradu"
%%% End:
