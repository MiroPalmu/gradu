\chapter{Electron Beam-Plasma Instabilities}

The beam–plasma system consists of two homogeneous,
counter-streaming pair-plasma populations.
In this chapter, we provide a brief overview of the instabilities
that arise in such a system, derive the maximum growth rates in the cold-fluid limit,
and show that these agree with the measured growth rates from symmetric
beam–plasma simulations performed with GPU-accelerated Runko.
In the cold-fluid limit and for symmetric beams,
the fastest-growing instability is the so-called filamentation instability,
whose unstable modes are fully electromagnetic~\cite{bret_collective_2004}.
Matching these growth rates therefore demonstrates the correctness of Runko’s
ability to resolve the full set of electromagnetic Vlasov–Maxwell equations.

\section{Overview}

Beam-plasma systems have been studied since Langmuir first suggested
the existence of oscillations in such systems in
1925~\cite{langmuir_scattering_1925,briggs_electron-stream_1964}.
Later, in 1948, Pierce explained Langmuir's observations by demonstrating
that unstable oscillations can arise in such systems \cite{pierce_possible_1948}.
This prompted Bohm and Gross to develop the kinetic theory of ''two-stream instability'',
which contains unstable waves propagating along the beam direction \cite{bohm_theory_1949}.
In 1959, Fried showed that a second kind of instability exists,
in which electromagnetic perturbations perpendicular to the flow
become unstable~\cite{fried_mechanism_1959}.
These modes tend to break up an initially homogeneous
beam profile into small-scale current filaments,
giving rise to the name ''filamentation instability''.
Fried's article also mentioned closely related work by Weibel,
who demonstrated the instability of
an anisotropic, two-temperature Maxwellian plasma~\cite{weibel_spontaneously_1959}.
In literature, these two instabilities have become almost interchangeable,
even though they are technically not equivalent~\cite{bret_multidimensional_2010}.
In this thesis, we use the term ''filamentation instability'' for clarity.

In addition to the parallel two-stream modes and perpendicular filamentation modes,
the stability of oblique modes can also be investigated.
This was proptly done in the cold-fluid limit, where it was found that
the unstable spectrum is two-dimensional,
as the oblique perturbations are likely to be unstable
\cite{watson_statistical_1960,bludman_statistical_1960,fainberg_nonlinear_1970}.
In 21st century, Bret et al. carried out comprehensive kinetic treatments
that provided a unified description of the entire unstable spectrum,
including temperature effects
\cite{bret_collective_2004,bret_electromagnetic_2005,bret_characterization_2005},
supported by numerous PIC simulations~\cite{gremillet_linear_2007,
  dieckmann_evolution_2006,
  frederiksen_electromagnetic_2008,
  kong_evolution_2009,
  karmakar_detailed_2009}.

\section{Growth rate in the cold-fluid limit}

As stated at the beginning of this chapter,
our system consists of two homogeneous,
counter-streaming pair-plasma populations.
Ions are treated as a fixed, neutralizing background.
We refer to the two populations as the beam and the background plasma,
and denote their number densities as $n_b$ and $n_p$, respectively.
Similarly, we denote their bulk velocities by $\ve V_b$ and $\ve V_p$.
Initially, the system is current- and charge-neutral:
\begin{align}
  \label{eq:current-neutrality}
  n_{b0}\ve V_{b0} = n_{p0}\ve V_{p0}
\end{align}
and without electromagnetic fields.
We define the beam to propagate along the x-axis.
Without loss of generality, we assume in the following discussion
that the wave vector lies in the xy-plane.

Following the symbolic-calculator technique from \cite{bret_beam-plasma_2007},
we solve for $\ve V_{b1}$ and $\ve V_{p1}$ from the linearized momentum transport equation
(see Equation~\eqref{eq:linearized-cold-fluid-momentum-transport})
and insert them to the linearized continuity equation
(see Equation~\eqref{eq:linearized-cold-fluid-continuity-eq})
to obtain $n_{b1}$ and $n_{p1}$.
These are then inserted into the linearized wave equation
(see Equation~\eqref{eq:linearized-wave-eq}).
The resulting expression can be rearranged into
a matrix form~\cite{bret_collective_2004,bret_cfa_2012}:
\begin{align}
  \label{eq:dispersion-eq-vec}
  \ve T(\ve k, \omega ) \hat{\ve E}_1 = 0
\end{align}
where $\ve T(\ve k, \omega ) \hat{\ve E}_1$ is a matrix-vector product between the matrix
$\ve T(\ve k, \omega )$ and the electric field perturbation $\hat{\ve E}_1$.
The matrix has the following form:
\begin{align}
  \ve T(\ve k, \omega ) = \frac{\omega^2}{c^2} \boldsymbol{\epsilon}(\ve k, \omega)
  + \ve k \otimes \ve k - k^2 \mathbb{I}
\end{align}
where $\mathbb{I}$ is the unit matrix and $\boldsymbol{\epsilon}(\ve k, \omega)$
is the dielectric tensor:
\begin{align}
  \ve T(\ve k, \omega) =
  \begin{bmatrix}
\frac{\omega^2}{c^2}\epsilon_{xx} - k_z^2 & 0 & \frac{\omega^2}{c^2}\epsilon_{xz} + k_z k_x \\
0 & \frac{\omega^2}{c^2}\epsilon_{yy} - k^2 & 0 \\
\frac{\omega^2}{c^2}\epsilon_{xz} + k_x k_z & 0 & \frac{\omega^2}{c^2}\epsilon_{zz} - k_x^2
\end{bmatrix}
\end{align}
Equation~\eqref{eq:dispersion-eq-vec} admits solutions only when the determinant of
$\ve T(\ve k, \omega)$ vanishes, which results in two branches:
\begin{align}
  \epsilon_{yy} = \frac{c^2k^2}{\omega^2}
\end{align}
and
\begin{align}
  \label{eq:dispersion-eq}
  \left( \frac{\omega^2}{c^2}\epsilon_{xx} - k_z^2 \right)
  \left( \frac{\omega^2}{c^2}\epsilon_{zz} - k_x^2 \right)
  = \left( \frac{\omega^2}{c^2}\epsilon_{xz} + k_x k_z \right)^2
\end{align}

For the instabilities, the first branch is not interesting, as the second holds
solutions corresponding to the two-stream, the oblique and the filamentation instabilities.
To investigate the nature of the solutions,
we first define standard dimensionless variables:
\begin{align}
  x = \frac{\omega}{\omega_{pp}} \qquad
  \ve Z = \frac{\ve k V_{b0}}{\omega_{pp}} \qquad
  \alpha = \frac{n_{b0}}{n_{p0}}
\end{align}
Now $x$ can be solved numerically from dispersion relation
shown in Equation~\eqref{eq:dispersion-eq}.
Figure~\ref{fig:perp-growth-rates} presents
the normalized growth rate (i.e. imaginary part of $x$)
as a function of $\ve Z$ for two different cases.
For the diluted case ($\alpha \ll 1$), we see that the largest growth rate
occurs in the oblique-angle regime, while for the symmetric case ($\alpha = 1$)
the largest growth is at $Z_z = 0$ and $Z_x \rightarrow \infty$,
corresponding to the filamentation instability.

\begin{figure}
  \centering
  \includegraphics{chapters/growth-rate.pdf}
  \caption{Growt rates from dispersion relation seen in Equation~\eqref{eq:dispersion-eq}
    for $\gamma_b = 3$.
    Left is the symmetric case $\alpha = 1$ and right is the diluted case $\alpha = 0.1$. }
  \label{fig:perp-growth-rates}
\end{figure}

For the transverse case $k_z = Z_z = 0$, the components of dielectric tensor
$\boldsymbol{\epsilon}(\ve k, \omega)$ simplify to~\cite{bret_cfa_2012}\footnote{
  Note that in \cite{bret_cfa_2012} powers of $-3$ are missing from Equation~\eqref{eq:mistake-a}
  and there is extra $\gamma_p^{-1}$ in Equation~\eqref{eq:mistake-b}.
}:
\begin{align}
  \epsilon_{xx} &= 1 - \frac{\alpha}{\gamma_b x^2} - \frac{1}{\gamma_p x^2}, \\
  \epsilon_{yy} &= 1 - \frac{\alpha}{\gamma_b x^2} - \frac{1}{\gamma_p x^2}, \\
  \label{eq:mistake-a}
  \epsilon_{zz} &= 1 + \frac{Z_x x^2 \alpha^2}{\gamma_p x^4} - \frac{Z_x x^2 \alpha}{\gamma_b x^4}
                  - \frac{\alpha}{\gamma_b^3 x^2} - \frac{1}{\gamma_p^3 x^2}, \\
  \label{eq:mistake-b}
  \epsilon_{xz} &= \frac{Z_x \alpha}{x^3} \left( \frac 1\gamma_p - \frac 1\gamma_b \right)
\end{align}
Inserting these into the dispersion relation (see Equation~\eqref{eq:dispersion-eq})
and taking the limit $Z_x \rightarrow \infty$, the growth rates for
symmetric and diluted cases of filamentation instability can be solved exactly:
\begin{align}
  \label{eq:filamentation-growth-rate-diluted}
  \Im (x) &= \frac{V_b}{c}\sqrt{\frac{\alpha}{\gamma_b}}, \quad \alpha \ll 1 \\
  \label{eq:filamentation-growth-rate-symmetric}
  &= \frac{V_b}{c}\sqrt{\frac{2}{\gamma_b}}, \quad \alpha = 1
\end{align}

\section{Numerical validation}

GPU-accelerated Runko was used to simulate filamentation instability
of symmetric beams ($\alpha = 1$) following the setup from~\cite{nattila_runko:_2022}.
The two counter-streaming pair-plasma populations are initialized to
flow along $\hat{\ve e}_0$ with spatial components of a four-velocity
$\ve u = \pm \gamma_b \hat{\ve e}_0$,
where $\gamma_b$ is the Lorentz factor shared between the symmetric beams.
The plasmas are modeled with 32 particles per cell per species,
which initial positions are sampled from a unifrom distribution
and velocities are sampled from Jüttner-Synge distribution~\cite{zenitani_loading_2015}
with thermal spread of $k_bT/(mc^2) = 10^{-5}$,
where $k_b$ is the Boltzmann constant, $T$ is the temperature,
$m$ is the particle mass, and $c$ is the speed of light.
Initially, the electromagnetic fields are set to zero.
The periodic simulation domain consists of a grid of $320\times80\times6$ uniform cells
and the skin depth is resolved with 10 cells.
The time step is related to the cell width as: $\Delta t = 0.45 \Delta x$.

For a comparison of the simulation results with the analytical growth rate of
Equation~\eqref{eq:filamentation-growth-rate-symmetric},
Figure~\ref{fig:perp-growth-rates} shows the average magnetic‑field energy density,
normalized by the relativistic plasma enthalpy density, as a function of time.
We note that the analytical growth rate matches the numerical growth rates perfectly
during the linear growth phase of the instability.
However, the linear phase does not begin immediately,
as the instability must first grow from numerical noise,
and it takes some time before it begins to dominate.
At a later stage, nonlinear effects terminate the linear growth phase,
and the growth saturates.
Qualitatively, the formed filaments begin to collide with each other,
disturbing the system sufficiently to halt further growth.

\begin{figure}
  \centering
  \includegraphics{chapters/instability-plot.pdf}
  \caption{Simulation of the filamentation instability using the GPU~accelerated Runko
    for different values of $\gamma_b$. Solid curves represent the averagemagnetic-field
    energy density $\rho_B = \frac{B^2}{8\pi}$,
    normalized by the relativistic plasma enthalpy density, $\gamma_bn_0m_ec^2$,
    as a function of time $t\omega_p$.
    From left to right, the values for $\gamma_b$
    are $3$ (blue), $10$ (orange), $30$ (green), and $100$ (red).
    The dashed lines show the analytic growth rates from the linearized cold-fluid limit.
  }
  \label{fig:perp-growth-rates}
\end{figure}

%%% local Variables:
%%% mode: LaTeX
%%% TeX-master: "../gradu"
%%% End:
