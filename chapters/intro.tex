\chapter{Introduction}
Plasma dynamics has been studied in many contexts, including astrophysics, laser physics, and fusion.
The dynamics are highly non-linear, and although there is a wealth of theoretical analysis,
the complexity eventually becomes infeasible to handle analytically.
Consequently, numerical methods are required to study many complex plasma phenomena.
Today, high-performance computing simulations are performed on supercomputers
to model the underlying physical systems with such accuracy that they can reveal new physical insights.
One commonly used approach is the Particle-in-Cell (PIC) method, in which particles are represented as computational particles.

This thesis is based on the work carried out by the author to port Particle-in-Cell (PIC) simulation
Runko~\cite{nattila_runko:_2022} to run on GPUs.
In this chapter, the required background theory is presented,
and in the following chapter, different aspects of Particle-in-Cell simulation are discussed.
In the third chapter, the GPU port of Runko is used to simulate
the electron beam–plasma filamentation instability, which is then compared to the analytical result
inorder to confirm physical accuracy of the port.
The final chapter contains a discussion of further development avenues for the GPU port of Runko.

\section{Plasma}

Plasma is a quasi-neutral gas containing enough free charges
that long-range collective electromagnetic interactions dominate
over interactions between individual particles~\cite{pert_foundations_2021,hazeltine_framework_2018}.
This can occur, for example, when a gas is heated sufficiently for its atoms to become ionized.
Quasi-neutrality means that the gas is approximately electrically neutral:
\begin{align}
  \rho = \sum_s n_s q_s \approx 0\,,
\end{align}
where the sum is taken over all particle species, $n_s$ denotes the number density
and $q_s$ is the charge of species $s$.

If a charge is introduced into a plasma in thermal equilibrium,
the free charges quickly screen the resulting long-range electric field.
This phenomenon is known as Debye screening (or Debye shielding).
It gives rise to a characteristic length scale over which short-range Coulomb interactions can occur.
This scale is defined by the Debye length, $\lambda_D$ which is given by:
\begin{align}
  \label{eq:debye-length}
  \lambda_D^{-2} = \sum_s \frac{n_sq_s^2}{\epsilon_0 k_B T_s}\,,
\end{align}
where $T_s$, $n_s$ and $q_s$ are the temperature, number density and charge of particle species $s$;
$k_B$ is the Boltzmann constant; and $\epsilon_0$ is the vacuum permittivity.

For a plasma in thermal equilibrium,
the scale of deviations from quasi-neutrality over a length $L$ due to thermal fluctuations is
$\left(\frac{\lambda_D}{L}\right)^2$~\cite{wiesemann_short_2014,hazeltine_framework_2018}.
Thus, we can express the quasi-neutrality condition for a plasma over the length scale $L$
as the requirement that the Debye length is much smaller than $L$:
\begin{align}
  \lambda_D \ll L\,.
\end{align}
However, for the Debye length to be statistically valid, there must be enough particles inside
the ''Debye sphere'': $N_D = n \frac 43 \pi \lambda_D^3 \gg 1$.
This requirement is usually expressed in terms of the plasma parameter,
which is defined (up to a constant factor) as $\Lambda = n\lambda_D^3$ and written as:
\begin{align}
  \label{eq:plasma-param-condition}
  \Lambda \gg 1\,.
\end{align}

\section{Fundamental frequencies in plasma physics}

There are a few characteristic frequencies that appear frequently in plasma physics.
In a plasma, a small displacement of charges oscillates around the neutral configuration at a frequency
known as the plasma frequency~\cite{wiesemann_short_2014,pert_foundations_2021,hazeltine_framework_2018}.
For a particle species $s$, the classical plasma frequency is given by:
\begin{align}
  \label{eq:classical-plasma-frequency}
  \omega_{ps}^2 = \frac{n_s q_s^2}{\epsilon_0 m_s}\,.
\end{align}
For electron–ion systems, the electron plasma frequency is much larger than the ion plasma frequency;
therefore, ion plasma oscillations can usually be neglected.

If a magnetic field is present in a plasma,
the charges gyrate around the field lines at the cyclotron frequency.
For a particle species $s$, the cyclotrone frequency is given by:
\begin{align}
  \label{eq:cyclotron-frequency}
  \omega_B = \frac{q_sB}{m_sc\gamma}\,,
\end{align}
where $B$ is the magnitude of the magnetic field.
This motion is also known as cyclotron motion.

\section{Kinetic plasma description}

As seen from condition~\eqref{eq:plasma-param-condition}, the number of particles in plasma is large.
This makes modeling a plasma
using only particles and Coulomb forces between them computationally unfeasible.
In the kinetic description of plasma,
the problem is addressed by embracing the large number of particles
and statistically describing their collective behavior.

For a particle species $s$, the distribution function $f_s(\ve x, \ve p, t)$ describes
the number of particles in a phase space region spanned by the time interval $[t, t + dt]$,
spatial region $[\ve x , \ve x + d\ve x]$ and three-momentum region
$[\ve p, \ve p + d\ve p]$~\cite{klimontovich_relativistic_1960,DeGroot:1980dk,hazeltine_framework_2018}.
The particles lie on a mass-shell defined by the Lorentz factor
$\gamma^{-1} = \sqrt{1-\frac{v^2}{c^2}}$ where $v$ is the coordinate velocity
and $c$ is the speed of light.
We assume that the distribution function is normalized such that
\begin{align}
  \label{eq:number-density}
  \int f_s(\ve x, \ve p, t) d\ve p = n_s(\ve x, t)\,,
\end{align}
where $n_s(\ve x, t)$ is the number density of a particle species $s$.
Summing over all particle species, we obtain the total charge density of the plasma:
\begin{align}
  \label{eq:charge-density}
  \rho(\ve x, t) = \sum_s q_s n_s(\ve x, t)\,.
\end{align}
Current density of the plasma is given by:
\begin{align}
  \label{eq:current-density}
  \ve J(\ve x, t) = \int \sum_s q_s \ve v f_s(\ve x, \ve p, t) d\ve p\,.
\end{align}
With these definitions of charge and current densities,
we can determine the self-consistent electromagnetic fields using Maxwell’s equations
(in Gaussian CGS units):
\begin{align}
  \label{eq:maxwells}
  \nabla\times\ve E = -\frac1c \frac{\partial\ve B}{\partial t}\,, \qquad
  \nabla\times\ve B = \frac{4\pi}c \ve J + \frac1c \frac{\partial\ve E}{\partial t}\,, \qquad
  \nabla\cdot\ve E = 4\pi\rho\,, \qquad
  \nabla\cdot\ve B = 0
\end{align}

The evolution of the distribution function is governed by the Boltzmann equation:
\begin{align}
  \label{eq:boltzmann}
  \partial_t f_s + \ve v \cdot \nabla f_s + \frac{d\ve p}{dt} \cdot \nabla_{\ve p} f_s
  = \left(\frac{\partial f_s}{\partial t}\right)_{\text{coll}}\,,
\end{align}
where $\left(\frac{\partial f_s}{\partial t}\right)_{\text{coll}}$ is the collision term,
which accounts for interactions between individual particles.
In the case of plasmas, $\frac{d\ve p}{dt}$ is given by the Lorentz force.
Due to condition \eqref{eq:plasma-param-condition},
short-range collisions between paritcles can be neglected, and the collision term vanishes.
When the collision term vanishes, the equation is called the Vlasov equation:
\begin{align}
  \label{eq:vlasov}
  \partial_t f_\alpha
  + \mathbf{v} \cdot \nabla f_\alpha
  + q_\alpha ( \mathbf{E} + \boldsymbol{\beta} \times \mathbf{B}) \cdot \nabla_\mathbf{p} f_\alpha = 0\,,
\end{align}
where we have defined the vector $\boldsymbol{\beta} = \frac{\ve v}{c}$.
Maxwell's equations together with the Vlasov equation form the Vlasov-Maxwell system.

\section{Cold-fluid limit}

Solving the Vlasov–Maxwell system is challenging.
Fortunately, in the case of beam instabilities,
taking the cold-fluid limit of the system allows us to investigate these instabilities analytically.
To do so, we calculate the moments of the Vlasov equation.

In general, the moment $M(\mathbf{p})$ of the Vlasov equation~\eqref{eq:vlasov} is:
\begin{align}
\int M \left(
\partial_t f_\alpha
+ \mathbf{v} \cdot \nabla f_\alpha
+ q_\alpha ( \mathbf{E} + \boldsymbol{\beta} \times \mathbf{B}) \cdot \nabla_\mathbf{p} f_\alpha
\right) d\mathbf{p} = 0\,,
\end{align}
where the integration is carried out over momentum space.
In the first two terms, the derivatives can be trivially taken outside the integrand,
but the third term requires some manipulation.

For brevity, we define $a_\alpha^i = q_\alpha E^i + \frac{q_\alpha}{c}\epsilon^i_{lk}v^l B^k$.
Using the identity $ \partial_{p_i} v^l = \frac{\delta^l_i}{m \gamma} -  \frac{p^lp_i}{m^3c^2\gamma^3}$
we find that $\partial_{p_i}a_\alpha^i = 0$.
After integration by parts and noting
that physical distributions vanish at infinity, we can write the third term as:
\begin{align}
\int M \mathbf{a_\alpha} \cdot \nabla_\mathbf{p} f_\alpha d\mathbf{p}
&= - \int a^i \partial_{p_i} M f d\mathbf{p}\,.
\end{align}
Overall we obtain:
\begin{align}\label{eq:general-moment-of-vlasov}
  \partial_t \int M f_\alpha d\mathbf{p}
+ \partial_i \int M v^i f_\alpha d\mathbf{p}
= \int a^i \partial_{p_i} M f_\alpha d\mathbf{p}\,.
\end{align}

A cold-fluid is characterized by a distribution function
$f(\mathbf{x}, \mathbf{p}, t) = n_\alpha(\ve x, t)\delta(\mathbf{p} - \mathbf{P}_\alpha(\mathbf{x}, t))$,
where $\mathbf{P}_\alpha(\mathbf{x}, t) = m_\alpha\gamma_\alpha\mathbf{V}_\alpha(\mathbf{x}, t)$
is calculated from the bulk velocity $\mathbf{V}_\alpha(\mathbf{x}, t)$ of the fluid,
and $n_\alpha(\mathbf{x}, t)$
is its fuilds number density~\cite{bret_multidimensional_2010,bret_collective_2004}.
Inserting this and $M = 1$ into equation~\eqref{eq:general-moment-of-vlasov},
we obtain the continuity equation~\cite{hazeltine_framework_2018}:
\begin{align}
\label{eq:continuity-in-cold-limit}
\partial_t n_\alpha + \nabla \cdot ( \ve V_\alpha n_\alpha ) = 0\,.
\end{align}
Using the continuity equation for $M = \mathbf{p}$, we obtain:
\begin{align}
\label{eq:momentum-transport-in-cold-limit}
\partial_t \mathbf{P}_\alpha + (\mathbf{V}_\alpha \cdot \nabla) \mathbf{P}_\alpha
&= q_\alpha (\mathbf{E} + c^{-1}\mathbf{V}_\alpha \times \mathbf{B})\,.
\end{align}

\section{Linearization of the cold-fluid limit}

Even in the cold-fluid limit,
solving for the general evolution of the system is challenging,
so we resort to solving it perturbatively to a certain order.
In this thesis, linear-order solutions to
equations~\eqref{eq:continuity-in-cold-limit} and~\eqref{eq:momentum-transport-in-cold-limit}
are sufficient. Furthermore, we assume that the initial electromagnetic fields vanish\footnote{
  The analysis could be done with non-zero initial magnetic field for which
  $\ve V_0 \times \ve B_0 = 0$. However, this is unnecessary in context of this thesis.
}.

Denoting equilibrium solutions with the subscript $0$ and small perturbations with the subscript $1$,
we can write (dropping particle species labels for brevity):
\begin{align}
  \label{eq:preturbed-variables}
  n &= n_0 + n_1\,, \\
  \ve V &= \ve V_0 + \ve V_1\,, \\
  \ve E &= \ve E_1\,, \\
  \ve B &= \ve B_1\,, \\
  \ve P &= m\gamma[\ve V_0 + \ve V_1](\ve V_0 + \ve V_1)\,,
\end{align}
where $\gamma[\ve V]$ denotes the Lorentz factor corresponding to the coordinate velocity $\ve V$.
We use the Fourier representation for the perturbations,
i.e., $g_1 = \int \hat{g}_1 \exp(i(\ve k \cdot \ve x - \omega t))d\omega d^3k$.
Keeping only the linear-order terms, the expression for the momentum simplifies to
$\ve P = m\gamma\left(\ve V_0+\frac{\gamma^2}{c^2}(\ve V_0 \cdot \hat{\ve V}_1)\ve V_0 \right)$,
where we defnie $\gamma \equiv \gamma[\ve V_0]$.

Inserting these into \eqref{eq:continuity-in-cold-limit} and \eqref{eq:momentum-transport-in-cold-limit},
we obtain the linearized continuity equation
and the linearized momentum transport equation~\cite{bret_collective_2004,bret_cfa_2012}:
\begin{align}
  \label{eq:linearized-cold-fluid-continuity-eq}
  \hat{n}_1 &= \frac{n_0 \ve k \cdot \hat{\ve V}_1}{\omega - \ve k \cdot \ve V_0}\,, \\
  \label{eq:linearized-cold-fluid-momentum-transport}
  im\gamma(\ve k \cdot \ve V_0 - \omega)
  (\hat{\ve V}_1 + \frac{\gamma^2}{c^2}(\ve V_0 \cdot \hat{\ve V}_1)\ve V_0)
  &= q(\hat{\ve E}_1 + \frac 1c \hat{\ve V}_1 \times \ve B_0 + \frac 1c \hat{\ve V}_0 \times \ve B_1)\,.
\end{align}
Additionally, we utilize linearity of Maxwell's equations~\eqref{eq:maxwells}
and combine them to obtain the linearized wave equation:
\begin{align}
  \label{eq:linearized-wave-eq}
  \ve k \times (\ve k \times \hat{\ve E}_1)
  + \frac{\omega^2}{c^2}\left(\hat{\ve E} _1 + \frac{4\pi i}{\omega}\hat{\ve J}_1\right) = 0\,,
\end{align}
where $\hat{\ve J}_1$ in the cold-fluid limit is read from:
\begin{align}
  \ve J_1 &= \int \sum_s q_s(n_{s0}\hat{\ve V}_{s1} + n_{s1}\hat{\ve V}_{s0})
  \exp(i(\ve k \cdot \ve x - \omega t))d\omega d^3k \\
  &= \int \hat{\ve J}_1 \exp(i(\ve k \cdot \ve x - \omega t))d\omega d^3k\,.
\end{align}

\section{Finite difference}
\label{sec:finite-difference}

The PIC scheme requires solving Maxwell's equations numerically on a discretized grid.
The most common way to discretize the derivatives is the centered finite-difference method,
which leads to finite-difference time domain method which is discussed in the next section.
Let the function $f(x)$ be discretized on a uniform grid of points $n h$,
where $n \in \mathbb{N}$ and $h$ is the constant grid spacing.
For brevity, we define the notation $f_n = f(n h)$
and $f'_n = \left.\frac{df}{dx}\right\vert_{x=nh}$.

The second-order accurate centered finite-difference discretizes the derivative as:
\begin{align}
  \label{eq:centered-fd}
  f'_n \approx \frac{f_{n + 1} - f_{n - 1}}{2h}\,.
\end{align}
The second-order accuracy of the algorithm is observed by Taylor expanding
the right-hand side of equation \eqref{eq:centered-fd}:
\begin{align}
  \frac{f_{n + 1} - f_{n - 1}}{2h}
  = \frac{1}{2h} \left(\sum_{k=0}^\infty \frac{h^k}{!k}f^{(k)}_n
  - \sum_{k=0}^\infty \frac{(-h)^k}{!k}f^{(k)}_n \right)
  = f'_n + O(h^2)\,.
\end{align}
A centered second-order derivative can be obtained by applying the centered first-order derivative twice:
\begin{align}
  \label{eq:second-fd}
  f''_n \approx \frac{f'_{n + 1} - f'_{n - 1}}{2h} = \frac{f_{n+2} - 2 f_n + f_{n-2}}{4h^2}\,.
\end{align}
Similarly to the first derivative, we can see the second-order accuracy from the Taylor expansion:
\begin{align}
  \label{eq:second-fd-accuracy}
  \frac{f_{n+2} - 2 f_n + f_{n-2}}{4h^2}  = f''_h + O(h^2)\,.
\end{align}
We note that the central finite-difference second-order derivative \eqref{eq:second-fd}
does not require the function values $f_{n+1}$ and $f_{n-1}$.
Thus, it is usually defined on a finer grid with grid points at $\frac{nh}{2}$, where $n \in \mathbb{N}$.
This gives us the standard centered finite-difference second-order derivative:
\begin{align}
  \label{eq:second-fd-correct}
  f''_n \approx \frac{f_{n+1} - 2 f_n + f_{n-1}}{h^2}\,,
\end{align}
which is still second-order accurate because the change only modifies
the constant coefficients in \eqref{eq:second-fd-accuracy}.

There exist higher-order accurate schemes for calculating the centered finite-difference,
but they involve a trade-off between improved numerical accuracy andincreased computational cost.

\subsection{Numerical dispersion relation}

The stability of the centered finite-difference scheme for the wave equation determines
its accuracy and stability properties. The one-dimensional wave equation is given by:
\begin{align}
  \label{eq:one-d-wave-eq}
  \frac{\partial^2 \psi}{\partial t^2} = c^2 \frac{\partial^2 \psi}{\partial x^2}\,.
\end{align}
Inserting the plane wave solution $\psi = e^{i(kx - \omega t)}$
to the wave equation~\eqref{eq:one-d-wave-eq} gives the physical dispersion relation:
\begin{align}
  \label{eq:one-d-wave-dispersion-eq}
  \omega = \pm ck\,.
\end{align}

We discretize the function $\psi(x, t)$ on a uniform grid $(n \Delta x, m \Delta t)$,
where $n,m\in\mathbb{N}$, and denote $\psi^m_n = \psi(n\Delta x, m\Delta t)$.
Applying the centered finite-difference for second derivatives~\eqref{eq:second-fd-correct}
to both the temporal and spatial derivatives in the wave equation~\eqref{eq:one-d-wave-eq},
and solving for $\psi_n^{m + 1}$, we obtain:
\begin{align}
  \label{eq:fd-intermediate}
  \psi_n^{m + 1} = \left(c\frac{\Delta t}{\Delta x}\right)^2
  (\psi_{n + 1}^m - 2\psi_n^m + \psi_{n-1}^m) + 2\psi_n^m - \psi_n^{m - 1}\,.
\end{align}
Inserting the discretized plane wave $\psi_n^m = e^{i(k n \Delta x - \omega m \Delta t)}$
into equation~\eqref{eq:fd-intermediate} results in a numerical dispersion relation
that is distinctly different from the physical dispersion relation~\eqref{eq:one-d-wave-dispersion-eq}:
\begin{align}
  \label{eq:fd-dispersion}
  \frac{\omega}{c} \Delta x =
  \pm \frac{1}{\hat c}\cos^{-1}\left(\hat{c}^2 (\cos(k\Delta x) - 1) + 1\right)\,,
\end{align}
where $\hat{c} \equiv c\frac{\Delta t}{\Delta x}$ is the Courant-Friedrichs-Lewy number,
or the Courant number for short.
The solutions are periodic in $k$ with a period of $\frac{2\pi}{\Delta x}$,
which means that the finite-difference wave equation~\eqref{eq:fd-intermediate}
cannot resolve waves beyond a certain wavenumber.

The numerical dispersion relation~\eqref{eq:fd-dispersion}
reduces to the physical dispersion relation~\eqref{eq:one-d-wave-dispersion-eq}
when $\hat{c} = 1$ or when both $\Delta t$ and $\Delta x$ approach zero,
but not when either one alone approaches zero.
These limits are summarized in table \ref{fig:numerical-dispersion-relation-limits}.

\begin{table}
  \centering
  \begin{tabular}{|c|c|}
    \hline
    & \textbf{Equation \eqref{eq:fd-dispersion}} \\ \hline
    $\hat{c} = 1$ & $\omega = \pm ck$ \\ \hline
    $\Delta t \rightarrow 0$ & $\omega \Delta x = \pm c \sqrt{2(1-\cos(k\Delta x))}$ \\ \hline
    $\Delta x \rightarrow 0$ & $\cos(\omega \Delta t) = -c^2k^2\frac{(\Delta t)^2}{2} + 1$ \\ \hline
    $\Delta t \rightarrow 0$ and $\Delta x \rightarrow 0$ & $\omega = \pm ck$ \\ \hline
  \end{tabular}
  \caption{Summary of different limits for the numerical dispersion relation~\eqref{eq:fd-dispersion}.}
  \label{fig:numerical-dispersion-relation-limits}
\end{table}

The Courant number $\hat{c}$ is crucial for the stability of solutions to
the finite-difference wave equation~\eqref{eq:fd-intermediate}.
For $\hat{c} > 1$, the numerical dispersion relation~\eqref{eq:fd-dispersion}
shows that $\omega$ has an positive imaginary part for some values of $k$,
which results in an exponentially growing mode.
The condition for stability, $\hat c = c\frac{\Delta t}{\Delta x} \le 1$,
is called the Courant-Friedrichs-Lewy condition.
In multiple dimensions, the condition is given by~\cite{verboncoeur_particle_2005}:
\begin{align}
  \Delta t \le \frac{1}{c}\left( \sum_i \frac{1}{(\Delta x_i)^2} \right)^{-1/2}\,.
\end{align}
Figure \ref{fig:cfl-plot} shows one period of
the numerical dispersion relation given in equation~\eqref{eq:fd-dispersion}.

\begin{figure}[h!]
  \centering
  \includegraphics[scale=0.55]{chapters/cfl-plot.pdf}
  \caption{Plots over one period of the numerical dispersion relation~\eqref{eq:fd-dispersion}
    for different values of the Courant number $\hat c = \frac{c\Delta t}{\Delta x}$.
    The real parts of $\frac{\omega}{c}\frac{\Delta x}{\pi}$ are shown as continuous lines,
    while the imaginary parts are shown as dashed lines.
    Courant numbers that satisfy the Courant-Friedrichs-Lewy condition are plotted
    with colors ranging from blue to green,
    whereas those that do not satisfy the condition are plotted in colors ranging from red to yellow.
    The black curve represents the transition point at $\hat c = 1$.
  }
  \label{fig:cfl-plot}
\end{figure}

%%% Local Variables:
%%% mode: LaTeX
%%% TeX-master: "../gradu"
%%% End:
