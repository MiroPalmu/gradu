\chapter{Particle-in-Cell}

For plasma simulation to include kinetic effects, it somehow has to be able to describe
evolution of the full distributions function. In PIC this is achieved by sampling
it with macroparticles then evolving them based on Maxwell's equations.
The electromagnetic fields are represented on a grid in which the particles move,
thus giving the name particle-in-cell.

PIC consists of many smaller tasks (e.g. pushing particles, evoling fields, etc.)
for which there exists numerous algorithms with differing trade-offs.
In this chapter we will describe the overall structure of PIC and go through
a set of algorithms implemented in the GPU port of Runko.

\section{Overall structure}
\section{Particle pusher}
\section{Field propagator}
\section{Field interpolator}
\section{Current depositer}
\section{Current filter}


%%% Local Variables:
%%% mode: LaTeX
%%% TeX-master: "../gradu"
%%% End:
