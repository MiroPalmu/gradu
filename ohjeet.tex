\section{Ohjeet}

Tässä pohjassa on kaikki, mitä yleiseen tutkielman kirjoittamiseen voisi
tarvita.

Tarvitset toimivan LaTeX-jakelun. Tämän lisäksi jotta lähdekoodiupotukset
toimivat, tarvitset \texttt{Pygments}-paketin, jonka voi asentaa esimerkiksi
käyttöjärjestelmäsi paketinhallintajärjestelmällä. Ubuntussa komento
\texttt{sudo apt install python-pygments} asentaa tarvittavan paketin. Jos et
halua tai voi asentaa Pygments-kirjastoa, voit myös antaa gradu-paketille vivun
\texttt{nominted}, jos pärjäät ilman \texttt{minted}-kirjastoa.

Tiedoston voi kääntää ohjelman~\ref{fig:compilelatex} mukaisella
komentorimpsulla. Voit myös käyttää tämän pohjan mukana tulevaa
\texttt{Makefile}-tiedostoa, jonka voi ajaa kirjoittamalla komentoriviin
\texttt{make}.

\begin{listing}[ht!]
    \inputminted{Shell}{compile.sh}
    \caption{Ohje tämän tiedoston kääntämiseen.}
    \label{fig:compilelatex}
\end{listing}

Korjaa tiedekuntasi ja laitoksesi \texttt{tktltiki.cls} -tiedoston riveille 258-272.

Laita lähteet \texttt{references.bib} -tiedostoon, jolloin niitä voi
käyttää \texttt{citep}-komennolla~\citep{latexpohja}.

Luo uusi \texttt{.tex}-tiedosto jokaiselle luvulle jotta tiedostot pysyvät
hallinnassa. Liitteitä voi myös käyttää, kuten liite~\ref{app:liite} näyttää.

Tässä pohjassa on kaikkea, mitä tutkielmaan voi tarvita ja vähän vielä päälle.
Jos haluat laittaa itsellesi todo-nootteja, \texttt{hl}-komentoa voi käyttää.
\hl{todo: parempi esimerkki}

Jalkanootteja saa \texttt{footnote}-komennolla\footnote{Jalkanootit tulevat
sivun pohjaan.}.

\newpage

\texttt{booktabs}-paketilla saa kauniita taulukoita:

\begin{table}[ht!]
    \begin{tabular}{@{}llp{8.5cm}@{}} \toprule
    Ladontaohjelma & Onko käyttökelpoinen? & Muistiinpanoja \\ \midrule
    \LaTeX         & Erittäin hyvä         & Your paper makes no goddamn sense, but it's the most beautiful thing I have ever laid eyes on. \\
    Word           & Ei ole                & Siirrä yhtä kuvaa 10 millimetriä vasemmalle. Dokumenttiin tulee kolme uutta sivua. Kaukaisuudessa kuuluu poliisiauton sireenit. \\ \bottomrule
    \end{tabular}

    \caption{LaTeX vastaan Word.}
    \label{table:properties}
\end{table}
